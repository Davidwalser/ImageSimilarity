%----------------------------------------------------------------
%
%  File    :  thesis.tex
%
%  Authors :  David Walser, Wien, Austria
% 
%  Created :  08 Feb 2022
% 
%  Changed :  08 Feb 2022
%
%  For suggestions and remarks write to: sebastian.ukleja@fh-campuswien.ac.at 
%----------------------------------------------------------------

% --- Setup for the document ------------------------------------

%Class for a book like style:
\documentclass[11pt,a4paper,oneside]{scrbook}
%For a more paper like style use this class instead:
%\documentclass[11pt,a4paper,oneside]{thesis}

%input encoding for windows in utf-8 needed for Ä,Ö,Ü etc..:
\usepackage[utf8]{inputenc}
%input encoding for linux:
%\usepackage[latin1]{inputenc}
%input encoding for mac:
%\usepackage[applemac]{inputenc}

\usepackage[ngerman]{babel}
%for english use this instead:
%\usepackage[english]{babel}

%needed for font encoding
\usepackage[T1]{fontenc}

% want Arial? uncomment next two lines...
%\usepackage{uarial}
%\renewcommand{\familydefault}{\sfdefault}

%some formatting packages
\usepackage[bf,sf]{subfigure}
\renewcommand{\subfigtopskip}{0mm}
\renewcommand{\subfigcapmargin}{0mm}

%For better font resolution in pdf files
\usepackage{lmodern}

\usepackage{url}

%\usepackage{latexsym}

\usepackage{geometry} % define pagesize in more detail

% --- Settings for header and footer ---------------------------------
\usepackage{scrlayer-scrpage}
\clearscrheadfoot
\pagestyle{scrheadings}
\automark{chapter}

%Left header shows chapter and chapter name, will not display on first chapter page use \ihead*{\leftmark} to show on every page
\ihead{\leftmark} 	
%\ohead*{\rightmark}	%optional right header
\ifoot*{Student*In}		%left footer shows student name
\ofoot*{\thepage}		%right footer shows pagination
%---------------------------------------------------------------------

\usepackage{colortbl} % define colored backgrounds for tables

\usepackage{courier} %for listings
\usepackage{listings} % nicer code formatting
\lstset{basicstyle=\ttfamily,breaklines=true}

\usepackage{graphicx}
  \pdfcompresslevel=9
  \pdfpageheight=297mm
  \pdfpagewidth=210mm
  \usepackage[         % hyperref should be last package loaded
    pdftex, 		   % needed for pdf compiling, DO NOT compile with LaTeX
    bookmarks,
    bookmarksnumbered,
    linktocpage,
    pagebackref,
    pdfview={Fit},
    pdfstartview={Fit},
    pdfpagemode=UseOutlines,                 % open bookmarks in Acrobat
  ]{hyperref}
\DeclareGraphicsExtensions{.pdf,.jpg,.png}
\usepackage{bookmark}

\usepackage[title]{appendix}

%paper format
\geometry{a4paper,left=30mm,right=25mm, top=30mm, bottom=30mm}

\setlength{\parskip}{3pt plus 1pt minus 0pt}       % vert. space before a paragraph

\setcounter{tocdepth}{1}        % lowest section level entered in ToC
\setcounter{secnumdepth}{2}     % lowest section level still numbered

%Start of your document beginning with title page
\begin{document}
\frontmatter

% --- Main Title Page ------------------------------------------------
\begin{titlepage}
\begin{picture}(50,50)
\put(-70,40){\hbox{\includegraphics{images/logo.png}}}
\end{picture}

\vspace*{-5.8cm}

\begin{center}

\vspace{6.2cm}

\hspace*{-1.0cm} {\LARGE \textbf{Bildähnlichkeitserkennung von Markenlogos mithilfe von Machine Learning \\}}
\vspace{0.2cm}
\hspace*{-1.0cm} Untertitel \\

\vspace{2.0cm}

\hspace*{-1.0cm} { \textbf{Masterarbeit\\}}

\vspace{0.65cm}

\hspace*{-1.0cm} Eingereicht in teilweiser Erfüllung der Anforderungen zur Erlangung des akademischen Grades: \\

\vspace{0.65cm}

\hspace*{-1.0cm} \textbf{Master of Science in Engineering\\}

\vspace{0.65cm}

\hspace*{-1.0cm} an der FH Campus Wien \\
\vspace{0.2cm}
\hspace*{-1.0cm} Studienfach: Software Design and Engineering \\

\vspace{1.6cm}

\hspace*{-1.0cm} \textbf{Autor:} \\
\vspace{0.2cm}
\hspace*{-1.0cm} David Walser \\

\vspace{0.7cm}

\hspace*{-1.0cm} \textbf{Matrikelnummer:}\\
\vspace{0.2cm}
\hspace*{-1.0cm} 01609388 \\

\vspace{0.7cm}

\hspace*{-1.0cm} \textbf{Betreuer:} \\
\vspace{0.2cm}
\hspace*{-1.0cm} FH-Prof. DI Dr. Igor Miladinovic \\

\vspace{0.7cm}

% Reviewer if needed:
%\hspace*{-1.0cm} \textbf{Reviewer: (optional)} \\
%\vspace{0.2cm}
%\hspace*{-1.0cm} Titel Vorname Nachname \\


\vspace{1.0cm}

\hspace*{-1.0cm} \textbf{Datum:} \\
\vspace{0.2cm}
\hspace*{-1.0cm} 02.02.2022 \\

\end{center}
\end{titlepage}

\newpage
\setcounter{page}{1}

\vspace*{16cm}

% --- Declaration of authorship ----------------------------------------------------
\hspace*{-0.7cm} \underline{Erklärung der Urheberschaft:}\\\\
Ich erkläre hiermit diese Masterarbeit eigenständig verfasst zu haben. Ich habe keine anderen Quellen, als die in der Arbeit gelisteten verwendet, noch habe ich jegliche unerlaubte Hilfe in Anspruch genommen\\\\
Ich versichere diese Masterarbeit in keinerlei Form jemandem Anderen oder einer anderen Institution zur Verfügung gestellt zu haben, weder in Österreich noch im Ausland.\\\\
Weiters versichere ich, dass jegliche Kopie (gedruckt oder digital) identisch ist.
\\\\\\
Datum: \hspace{6cm} Unterschrift:\\

% --- English Abstract ----------------------------------------------------
\cleardoublepage
\chapter*{Abstract}
(E.g. ``This thesis investigates...'')


% --- German Abstract ----------------------------------------------------

\cleardoublepage
\chapter*{Kurzfassung}
(Z.B. ``Diese Arbeit untersucht...'')

% --- Abbrevations ----------------------------------------------------
\newpage\noindent
\chapter*{Abkürzungen}
\vspace{0.65cm}

\begin{table*}[htbp]
		\begin{tabular}{ll}
			ÖPA & Österreichisches Patentamt \\
			ARP & Address Resolution Protocol \\
			GPRS & General Packet Radio Service \\
			GSM  &  Global System for Mobile communication \\
			WLAN & Wireless Local Area Network \\
		\end{tabular}
\end{table*}

% --- Key terms ----------------------------------------------------
\newpage
\chapter*{Schlüsselbegriffe}
\vspace{0.65cm}

\begin{itemize}
	\setlength{\itemsep}{0pt}
	\item[] GSM
	\item[] Mobilfunk
	\item[] Zugriffsverfahren
\end{itemize}

% --- Table of contents autogenerated ------------------------------------
\newpage
\tableofcontents
\thispagestyle{empty}

% --- Begin of Thesis ----------------------------------------------------
\mainmatter
\chapter{Einführung}
\label{chap:intro}
\section{Problembeschreibung}
\label{sec:Problembeschreibung}
Für uns Menschen ist es eine ziemlich einfache Aufgabe zu ermitteln ob ein Bild ähnlich zu einem anderen ist oder nicht. Wir erkennen 

\chapter{Einführung}
\label{chap:intro}

Textkörper mit Bild

\begin{figure}[htbp]
	\centering
		\includegraphics[height=5cm]{images/buecher.png}
	\caption{Ein Stapel Bücher}
	\label{fig:buecher}
\end{figure}


Textkörper Fortsetzung mit Verweis auf den wundervollen Stapel Bücher in Abbildung \ref{fig:buecher}. 


\section{Unterkapitel 1}
\label{sec:Unterkapitel1}

Textkörper mit Formel:

\begin{equation}
U(j\omega)=\int^{\infty}_{-\infty}{u(t) \cdot e^{-j\omega t}dt}
\label{form:form1}
\end{equation}

Textkörper Fortsetzung mit Verweis auf Formel \ref{form:form1}. Und nicht zu vergessen: es gibt auch noch eine tolle Abbildung in Kapitel \ref{chap:intro}, nämlich Abbildung \ref{fig:buecher}. 


\subsection{Unter-Unterkapitel 11}
\label{sec:UnterUnterkapitel11}

Textkörper mit direktem Zitat und Seitenanzahl:
``It would be very easy to show how technical or report writing differed from other writing'' \cite[p.~3]{young2002technical}.

\subsection{Unter-Unterkapitel 12}
\label{sec:UnterUnterkapitel12}

Textkörper mit Referenzen:
Für weiterführende Informationen zum wissenschaftlichen Schreiben siehe "J. Schimel, Writing Science" \cite{schimel2012writing}. Es wird empfohlen den Sprachleitfaden der FH Campus Wien \cite{alker2006} zu berücksichtigen und die Checkliste für wissenschafltiches Schreiben \cite{petz2018} zu verwenden. Beide Leitfäden sind im FH Portal zu finden.

\chapter{Kapitel 2}
\label{chap:back}

Textkörper mit noch einem Bild

\begin{figure}[htbp]
	\centering
		\includegraphics{images/birne}
	\caption{Eine Glühbirne}
	\label{fig:birne}
\end{figure}



\section{Unterkapitel 21}
\label{sec:Unterkapitel21}

Textkörper mit Tabelle.

\begin{table*}[htbp]
	\centering
		\begin{tabular}{|l|c|r|}
		\hline
		\rowcolor[gray]{0.9}
		Spalte 1 & Spalte 2 & Spalte 3 \\
		\hline
		Affen & Giraffen & Löwen \\
		Apfel & Birnen & Bananen \\
		Irgend & et & was \\
		\hline	
		\end{tabular}
	\caption{Beispiel für eine Tabelle}
	\label{tab:BeispielFuerEineTabelle}
\end{table*}

Man beachte die Gegenüberstellung in Tabelle \ref{tab:BeispielFuerEineTabelle}.

%Online Tabellengenerator für Latex: https://www.tablesgenerator.com/

\section{Unterkapitel 23}
\label{sec:Unterkapite23}

Aufzählungen:

Nummeriert:

\begin{enumerate}
	\item Punkt 1
	\item Punkt 2
\end{enumerate}

Mit Bullet Points:

\begin{itemize}
	\item Punkt 1
	\item Punkt 2
\end{itemize}

Mit Beschreibungen:

\begin{description}
	\item[Item 1] das ist der 1.Punkt
	\item[Item 2] und das der 2.
\end{description}


Auch Programmcodes können an entsprechender Stelle eingefügt werden, man beachte dazu auch Listing \ref{lst:conv}.

% see also http://mirror.easyname.at/ctan/macros/latex/contrib/listings/listings.pdf for options

\begin{lstlisting}[frame=lines, caption=Simple Listing, captionpos=b, label = lst:conv, language=C, showstringspaces=false]
#include <stdio.h>
int main()
{
	int i, n, t1 = 0, t2 = 1, nextTerm;

	printf("Enter the number of terms: ");
	scanf("%d", &n);

	printf("Fibonacci Series: ");

	for (i = 1; i <= n; ++i)
	{
		printf("%d, ", t1);
		nextTerm = t1 + t2;
		t1 = t2;
		t2 = nextTerm;
	}
	return 0;
}
\end{lstlisting}

Und zuguterletzt, Formeln mitten im Fliesstext, wie z.B. $a^2+b^2=c^2$, in einem Absatz.

\newpage
\chapter{Related Work}

\newpage
\chapter{Zusammenfassung}

\newpage
\chapter{Ausblick}

\newpage

% --- Bibliography ------------------------------------------------------

%IEEE Citation [1]
\bibliographystyle{IEEEtran}
%for alphanumeric citation eg.: [ABC19]
%\bibliographystyle{alpha}

% List references I definitely want in the bibliography,
% regardless of whether or not I cite them in the thesis.

\newpage
\addcontentsline{toc}{chapter}{Bibliographie}
\bibliography{testBib}

\newpage

% --- List of Figures ----------------------------------------------------

\addcontentsline{toc}{chapter}{Abbildungen}
\listoffigures


% --- List of Tables -----------------------------------------------------

\newpage
\addcontentsline{toc}{chapter}{Tabellen}
\listoftables

% --- Appendix A -----------------------------------------------------

\newpage
\appendix
\backmatter
\begin{appendices}
\chapter{Appendix}

(Hier können Schaltpläne, Programme usw. eingefügt werden.)

\clearpage
\end{appendices}

\end{document}
